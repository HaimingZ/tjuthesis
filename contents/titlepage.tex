% 扉页,可根据实际情况修改`\vsapce'值以调整各部分的垂直距离,使内容保持在一页之内

\thispagestyle{empty} % 去除页眉页脚
\begin{center} % 本页内容居中

\vspace*{0cm}

\zihao{2}\textbf{中文题目}

\vspace{0.8cm}

\zihao{4}\textbf{English Title}

\vspace{4cm}

\zihao{-4} % 切换为小四字号绘制下面两个表格

% 基本信息表
\begin{table}[h] 
    \centering 
    \renewcommand*{\arraystretch}{1.35} % 设置行高
    \begin{tabular}{cc}
        \textbf{一级学科:} & \textbf{\underline{\makebox[14em][c]{在此输入}}} \\
        \textbf{研究方向:} & \textbf{\underline{\makebox[14em][c]{输入于此}}} \\
        \textbf{作者姓名:} & \textbf{\underline{\makebox[14em][c]{某某某}}} \\
        \textbf{指导教师:} & \textbf{\underline{\makebox[14em][c]{某某某}}} \\
    \end{tabular}
\end{table}

% 答辩信息表,可根据实际答辩委员数增加行数,应修改 \multirow 的第一个参数
\begin{table}[h] 
    \centering 
    \renewcommand*{\arraystretch}{1.35} % 设置行高
    \begin{tabularx}{\textwidth}{|*{4}{>{\centering\arraybackslash}X|}}
    \hline
    \textbf{答辩日期}               & \multicolumn{3}{c|}{2022年11月29日}               \\ \hline
    \textbf{答辩委员会}             & \textbf{姓名}  & \textbf{职称} & \textbf{工作单位}  \\ \hline
    \textbf{主席}                  &               &               &                  \\ \hline
    \multirow{6}{*}{\textbf{委员}} &               &               &                  \\ \cline{2-4} 
                                  &               &               &                  \\ \cline{2-4}
                                  &               &               &                  \\ \cline{2-4}
                                  &               &               &                  \\ \cline{2-4}
                                  &               &               &                  \\ \cline{2-4}
                                  &               &               &                  \\ \hline
    \end{tabularx}
\end{table}

\vspace{0cm}

\zihao{4}{天津大学某某学院\\二〇二二年十一月}

\end{center} % 停止居中
