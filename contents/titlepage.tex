% 扉页

\pagestyle{empty} % 去除页眉页脚

\begin{center} % 本页内容居中

\vspace*{0.8cm} % 根据需要调整距离

\zihao{2}\textbf{中文题目(宋体二号加粗)}

\vspace{0.8cm} % 根据需要调整距离

\zihao{4}\textbf{English Title(Times New Roman, 14-22 pt., Bold)}

\vfill

\zihao{-4} % 切换为小四字号绘制下面两个表格

% 基本信息表
\begin{table}[h] 
    \centering 
    \renewcommand*{\arraystretch}{1.35} % 设置行高
    \begin{tabular}{cc}
        \textbf{一级学科:} & \textbf{\underline{\makebox[14em][c]{输入}}} \\
        \textbf{研究方向:} & \textbf{\underline{\makebox[14em][c]{输入}}} \\
        \textbf{作者姓名:} & \textbf{\underline{\makebox[14em][c]{输入}}} \\
        \textbf{指导教师:} & \textbf{\underline{\makebox[14em][c]{输入}}} \\
    \end{tabular}
\end{table}

\vspace{0.5cm}  % 根据需要调整距离

% 答辩委员会名单表,可根据实际答辩委员数增加行数,注意同时修改 \multirow 的第一个参数
\begin{table}[h] 
    \centering 
    \renewcommand*{\arraystretch}{1.35} % 设置行高
    \begin{tabular}{|C{0.17\textwidth}|C{0.12\textwidth}|C{0.12\textwidth}|C{0.45\textwidth}|}
    \hline
    \textbf{答辩日期}              & \multicolumn{3}{c|}{20XX年XX月XX日} \\ \hline
    \textbf{答辩委员会}            &  \textbf{姓名} &  \textbf{职称} &  \textbf{工作单位}  \\ \hline
    \textbf{主席}                  &                &                &                     \\ \hline
    \multirow{6}{*}{\textbf{委员}} &                &                &                     \\ \cline{2-4} 
                                   &                &                &                     \\ \cline{2-4}
                                   &                &                &                     \\ \cline{2-4}
                                   &                &                &                     \\ \cline{2-4}
                                   &                &                &                     \\ \cline{2-4}
                                   &                &                &                     \\ \hline
    \end{tabular}
\end{table}

\vspace{0.5cm}  % 根据需要调整距离

\zihao{4}{天津大学XXXX学院\\二〇XX年XX月}

\vspace{0.5cm}  % 根据需要调整距离

\end{center} % 停止居中
